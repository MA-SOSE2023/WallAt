Wir, das Entwicklerteam bestehend aus Marcel Biselli, Lars Bürger und Simon Rauch, haben im Rahmen des AIN Kurses "Mobile Anwendungen" 
im Sommersemester 2023 eine mobile App namens "Wall@" entwickelt. 
Unser Ziel war es, eine innovative Lösung für das Scannen und Verwalten von Dokumenten zu schaffen.
\newline
\newline
Mit Wall@ bieten wir eine benutzerfreundliche und effiziente Möglichkeit, Dokumente digital zu erfassen und in einem sicheren digitalen Geldbeutel zu verwalten. 
Die App richtet sich an alle, die eine bequeme Methode suchen, um Dokumente wie Quittungen, Rechnungen, Ausweise oder andere wichtige Unterlagen zu scannen, 
zu speichern und jederzeit griffbereit zu haben.
\newline
\newline
Bei der Entwicklung von Wall@ haben wir bewährte Software-Engineering-Praktiken und -Standards berücksichtigt, 
um eine stabile, skalierbare und gut strukturierte App zu gewährleisten. 
Durch den Einsatz der Programmiersprache Dart und des Flutter-Frameworks ist es uns gelungen, eine plattformübergreifende Lösung zu entwickeln, 
die gleichermaßen auf den Betriebssystemen iOS und Android funktioniert.
\newline
\newline
Diese Dokumentation dient als verbindliche Ressource für Entwickler, die an der App weiterarbeiten möchten, 
und bietet Einblicke in die Implementierungsdetails sowie die zugrunde liegenden Technologien.
\newline
\newline
Bei weiteren Fragen, Feedback oder Anregungen stehen wir Ihnen gerne zur Verfügung. 
Wir hoffen, dass diese Dokumentation einen wertvollen Beitrag zur effizienten Nutzung und Weiterentwicklung von Wall@ leistet.