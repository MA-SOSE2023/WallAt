
Dieses Kapitel bietet einen Überblick über den Entwicklungsprozess von Wall@. 
Es behandelt die geplanten User Stories und den Einsatz von GitLab für das Projektmanagement und die Zusammenarbeit im Team.
Die User Stories dienen als Leitfaden für die Entwicklung und beschreiben die Funktionalität von Wall@ aus Benutzersicht. 
Der Entwicklungsprozess umfasst verschiedene Phasen und Praktiken, die sicherstellen, dass die App erfolgreich entwickelt, getestet und bereitgestellt werden kann.
Zusätzlich wird der Einsatz von GitLab vorgestellt, einer kollaborativen Entwicklungsplattform, die Funktionen zur Versionskontrolle und zum Projektmanagement bietet.

\section{Issues in GitLab}
Für die Koordination und Verwaltung des Entwicklungsprozesses haben wir das Ticket-System in GitLab genutzt. 
Mithilfe dieses Systems konnten wir Tickets erstellen, Aufgaben zuweisen und die Entwicklung effektiv koordinieren. 
Wir haben Funktionen wie Ticketzuweisung (an Verantwortliche), Labeling und Kommentarfunktionen genutzt, 
um die Zusammenarbeit und den Fortschritt innerhalb des Teams zu erleichtern.

\section{User Stories}
Aufgrund der Anforderungen an die App haben wir verschiedene User Stories erstellt, 
die die Funktionalität von Wall@ aus Benutzersicht beschreiben.
Eine Liste der User Stories spowie der aktuelle Entwicklungsstand der einzelnen Stories ist über ein 
seperates GitLab Board (siehe \href{https://gitlab.in.htwg-konstanz.de/mobile-anwendungen-ss23/gruppen/gruppe-4/-/boards/272}{User Stories Board}) einzusehen.

\section{GitFlow}
Für eine effiziente Verwaltung des Entwicklungsprozesses haben wir das GitFlow-Modell implementiert. 
Es ermöglichte uns, den Code in verschiedenen Branches zu organisieren und stabile sowie entwicklungsorientierte Umgebungen aufrechtzuerhalten. 
Wir haben den "master"-Branch für stabile Versionen, den "develop"-Branch für die Hauptentwicklung und Feature-Branches für neue 
Funktionen verwendet. Zusätzlich nutzten wir Hotfix-Branches, um kritische Fehler schnell zu beheben und in den "master" und 
"develop"-Branch zu überführen. Zudem wurden alle Commits mit einer entsprechenden Issue-Nummer versehen, 
um die Nachvollziehbarkeit zu gewährleisten.
