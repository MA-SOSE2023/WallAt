\section{Kamera via Package}

Bei der Entwicklung von Wall@ war es erforderlich, ein Kamera-Paket in die App zu integrieren, um die Funktionalität des Dokumentenscanners zu ermöglichen. 
Nach einer gründlichen Evaluierung wurden die folgenden drei verschiedene Kamera-Pakete untersucht, um die Anforderungen unserer App zu erfüllen.
\newline
\newline
\begin{tabular}{|l|>{\raggedright\arraybackslash}p{0.2\linewidth}|>{\raggedright\arraybackslash}p{0.3\linewidth}|}
  \hline
  Package & Feature & Problem \\
  \hline
  \href{https://pub.dev/packages/flutter_document_scanner}{flutter\_document\_scanner} & Auto Crop, Manuel Crop, ScannLook & compiling errors \\
  \hline
  \href{https://pub.dev/documentation/document_scanner_flutter/latest/}{document\_scanner\_flutter} & Auto Crop, Manuel Crop, ScannLook, pdf conv. & depends on photo\_view package 0.12.0 \\
  \hline
  \href{https://pub.dev/packages/cunning_document_scanner}{cunning\_document\_scanner} & Auto Crop, Manuel Crop & - \\
  \hline
\end{tabular}
\newline
\newline
\newline
Ursprünglich hatten wir uns für das zweite Kamera-Paket entschieden, das unsere Anforderungen optimal erfüllte. 
Jedoch ist dieses Paket nicht mit Dart 3 kompatibel, was zu Kompatibilitätsproblemen und potenziellen Einschränkungen 
bei der Weiterentwicklung führen könnte.
\newline
\newline
Um dieses Problem zu lösen, wurde ein Issue auf dem Entwickler-Git des bevorzugten Kamera-Pakets erstellt 
(siehe \href{https://github.com/ishaquehassan/document_scanner_flutter/issues/33}{Issue on GitHub}) und das Problem kommuniziert. 
In der Zwischenzeit haben wir uns für ein alternatives Kamera-Paket entschieden, das mit Dart 3 kompatibel ist und 
unsere grundlegenden Anforderungen erfüllt.
Durch die Nutzung des letzten Packet wurde die Funktionalität des Dokumentenscanners in Wall@ implementiert.
Es ist somit möglich Bilder mittels der Kamera von Dokumenten aufzunehmen wobei die Ränder automatisch erkannt werden 
und das Bild entsprechend zugeschnitten wird.
\newline
\newline
Langfristig planen wir jedoch, eine eigene Lösung für den Dokumentenscanner in Wall@ zu entwickeln. 
Dies ermöglicht uns eine maßgeschneiderte Implementierung, die besser auf die spezifischen Anforderungen und 
zukünftigen Erweiterungen unserer App abgestimmt ist. 
Die Entwicklung einer eigenen Lösung wird es uns auch ermöglichen, 
volle Kontrolle über die Funktionalität und die Integration in den Rest der App zu haben.
\newline
\newline
In diesem Zuge soll auch die Funktionalität des Dokumentenscanners erweitert werden, so dass Bilder nicht nur als 
Bild, sondern auch als PDF gespeichert werden können und diese mittels Filter bearbeitet werden können so dass 
z.B.: Probleme mit der Belichtung oder dem Kontrast behoben werden können.