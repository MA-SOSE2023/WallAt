\section{Cupertino-Widgets}
Die App verwendet vorwiegend Cupertino-Widgets, um eine Benutzeroberfläche zu erstellen, 
die den Design-Mustern von Apple-Anwendungen ähnelt.

\section{Adaptives Design mit MediaQuery und Slivers}
Um die App an verschiedene Bildschirmgrößen anzupassen, verwenden wir die MediaQuery-Klasse von Flutter.

Mit MediaQuery kann beispielsweise die verfügbare Bildschirmgröße ermittelt werden. 
Die App ist hiermit \textit{adaptiv} und einzelne Elemente werden in ihrer Breite und Höhe an den Bildschirm angepasst.
Das macht sie allerdings nicht \textit{responsive} und die App ist derzeit nur für mobile Endgeräte im Hochformat ausgelegt.

Zusätzlich zur MediaQuery nutzen wir Slivers, um die App adaptiver zu gestalten. 
Slivers ermöglichen es, flexible und scrollbare Layouts zu erstellen, die sich automatisch an den verfügbaren Bildschirmplatz anpassen.

\section{Animationen}

\subsection{Hero-Animation von Bildcontainern}
Preview Bilder für Dokumente sind in ein \verb|Hero|-Widget gewickelt. Dies erlaubt für einfache Animation des Bilds bei Navigation zu einem anderen Screen, auf dem das Bild ebenfalls vorhanden ist. 
Wenn ein Benutzer einen Eintrag in einer Liste antippt und in einen Einzeleintrag navigiert oder in der Einzelansicht das Bild im Vollbildmodus betrachtet, 
wird das Bild nahtlos von einem Container zum anderen animiert.

\subsection{Animation des Kamera-Buttons}
In der Home-Ansicht schwebt der Kamera-Button über der Navigationsleiste. In den anderen Seiten befindet sich dieser jedoch in der Leiste.
Deswegen nutzt dieser eine individuelle Animation, um die neue Position anzuzeigen.
Beim Navigieren, bewegt sich dieser in die Navigationsleiste hinein und wird letztlich transparent. Beim zurücknavigieren läuft die Animation verkehrt ab.
Um dies zu implementieren wird das selbst definierte \verb|CameraButtonHeroDestination|-Widget genutzt und auf den Seiten platziert, auf denen der Button in der Leiste zu sehen ist. Dort ist das Kamera-Icon in einem Hero-Widget mit einem individuellen \verb|flightShuttleBuilder| platziert, welcher eine \verb|FadeTransition| von opak zu transparent durchführt.

\section{Widgets der Woche}
Innerhalb des Projekts wurden eine Reihe von Flutter-Widgets verwendet, die als "Widgets der Woche" bekannt sind.
Hier sind einige der Widgets, die genutzt werden; drei davon mit näher beschriebenem Nutzen:

\begin{enumerate}
\item \textbf{CupertinoSliverNavigationBar:} Dieser bietet eine elegante und visuell ansprechende Navigationsleiste im iOS-Stil, welche den Titel der aktuellen Seite, sowie einen 'Zurück'-Knopf und weitere Steuer-Elemente enthalten kann

\item \textbf{HeroMode:} Da die Hero-Animation der Preview-Bilder zwischen dem Home-, Favorites-, und Folder-Screen unerwünscht und nur bei Navigation zur Detail-Ansicht der Einträge ablaufen soll, wird \verb|HeroMode| genutzt, um die Animation wenn nötig zu unterdrücken.

\item \textbf{FutureBuilder:} Mit diesem Widget macht asynchronen Zugriff auf Daten sehr viel einfacher. In Abschnitt \ref{sec:futures} wird dessen Nutzung näher beschrieben.
\end{enumerate}

\begin{tabular}{lll}
SafeArea & LinearGradient & StatefulBuilder \\
GestureDetector & LayoutBuilder & Hero\\
SliverAppBar & CupertinoActivityIndicator & Divider \\
CupertinoActionSheet & DraggableScrollableSheet & CupertinoAlertDialog \\
Stack & AnimatedOpacity & MediaQuery \\
Flexible & Dismissible & SizedBox \\
SliverList & SliverGrid & \\

\end{tabular}
\newline 

\section{Themes}

Unser Projekt implementiert ein flexibles Theme-System, das es uns ermöglicht, 
das Design unserer App einfach anzupassen oder neue Themes hinzuzufügen. 
Die Hauptfarben für sowohl das Dark- als auch das Light-Theme sind bewusst so gestaltet, 
dass sie den Apple-Farben ähneln jedoch unserem Color-CI entsprechen. 
Dadurch schaffen wir eine konsistente visuelle Ästhetik und sorgen dafür, 
dass unsere App in das Apple-Ökosystem passt.


\section{Spezifische Packages}
Wir nutzen verschiedene Packages in unserem Projekt, um spezifische Funktionalitäten zu implementieren. 
Dabei haben folgende Packages eine wichtige Rolle gespielt:

\begin{enumerate}
\item \textbf{social\_share:} Das social\_share Package wird genutzt, um den Benutzern die Möglichkeit zu geben, Einträge einfach und schnell über verschiedene soziale Medienplattformen zu teilen.

\item \textbf{device\_calendar:} Das device\_calendar Package spielte eine wichtige Rolle bei der Integration von Kalenderfunktionen. Mit diesem Package können Ereignisse zu den Systemkalendern der Benutzer hinzugefügt und entfert werden.

\item \textbf{isar:} Isar wurde verwendet, um eine effiziente Datenbank- und Persistenzverwaltung zu ermöglichen.

\item \textbf{beamer:} Beamer spielt eine wichtige Rolle bei der Implementierung des Routings innerhalb der App. Damit auch die komplexere und verschachtelte Navigation mit einer \verb|BottomNavBar| umgesetzt.

\end{enumerate}

\section{Splash Screen}
Neben dem Logo (siehe \ref{fig:logo}) wurde zudem ein ansprechender Splash Screen eingebaut, 
um einen visuell ansprechenden Start zu bieten, währenddessen die Datenbank geladen werden kann.